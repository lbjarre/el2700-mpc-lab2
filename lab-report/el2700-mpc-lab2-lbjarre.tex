\documentclass{article}

\usepackage{geometry}
\usepackage{fontspec}
\usepackage{mathtools}
\usepackage{unicode-math}
\usepackage{cleveref}
\usepackage{hyperref}

\geometry{
  vmargin={2cm, 2cm},
  hmargin={3cm, 3cm},
  a4paper
}

\setmainfont{texgyrepagella}[
  UprightFont=*-regular,
  ItalicFont=*-italic,
  BoldFont=*-bold,
  BoldItalicFont=*-bolditalic
]
\setmathfont{texgyrepagella-math}

\newcommand{\mail}[1]{\href{mailto:#1}{#1}}

\begin{document}

\begin{center}
  \textbf{
    \LARGE LAB2 \\
    \large EL2700 - Model Predictive Control \\
    \vspace{0.5cm}
    \begin{tabular}{c}
      Lukas Bjarre \\
      \mail{lbjarre@kth.se} \\
      921008-0256
    \end{tabular} \\
    \vspace{0.5cm}
    \rule{0.75\textwidth}{0.4pt}
  }
\end{center}

\section{Vehicle model}
The first step of designing a good MPC is to obtain a good model of the system which is to be controlled.

\subsection{Continous time system}
The continous time model used will be one derived by the bicycle model of the car. Given the internal states $z(t) = [x(t),\,y(t),\,v(t),\,\psi(t)]^T$ and the inputs $u(t) = [a(t),\,\beta(t)]^T$, the system dynamics can be expressed on the form $\dot{z}(t)=f(z(t),\,u(t))$ as:
\begin{equation}
  \underbrace{\dot{\begin{bmatrix}
    x(t) \\
    y(t) \\
    v(t) \\
    \psi(t) \\
  \end{bmatrix}}}_{\dot{z}(t)} = \underbrace{\begin{bmatrix}
    v(t)\cos\left(\psi(t) + \beta(t)\right) \\
    v(t)\sin\left(\psi(t) + \beta(t)\right) \\
    a(t) \\
    \frac{v(t)}{l_r}\sin\left(\beta(t)\right) \\
  \end{bmatrix}}_{f(z(t),\,u(t))}
  \label{eq:ctimemodel}
\end{equation}

\subsection{Discrete time system}
In order to implement the MPC the model needs to be discretizised. This is here done by the explicit Euler method. Introducing the sampling time $T_s$ and the discrete time variable $k=T_st$ the discretization yield the system:
\begin{equation}
  z(k+1) = z(k) + T_sf(z(k),\,u(k))
\end{equation}


\end{document}
